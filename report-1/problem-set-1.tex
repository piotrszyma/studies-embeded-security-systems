\section{Problem set 1}
\subsection{Preface}

The goal of this problem set was a brief introduction to ,,VHDL'', a hardware
description language used in electronic design automation.

\subsection{Assignment 1}

This assignment consisted of a typical introduction task - writing
,,Hello world'' program using ,,VHDL'' as it was described in ,,GHDL Quick Start
Guide'' attached to assignment description. In this task I've interacted
with first lines of code in ,,VHDL'', I got to know basic structure of program
- not yet fully understood - and how does the flow of \textit{running} the
program looks like.

\subsection{Assignment 2}

In this task I've extended program provided in ,,Quick Start Guide'' using
functionality provided by \texttt{textio} library. The goal of this task was to
create program that interacts with \texttt{stdio} - reads a line and prints it
to the console.

\subsection{Assignment 3}

Here I've learned what are the basic ,,VHDL'' entities - what is required for
a valid ,,VHDL'' program:
\begin{enumerate}
  \item entity - the most basic building block in ,,VHDL''.
  \item architecture - description of entity's behaviour.
  \item port - definition of in and out connections to the entity.
  \item component - in the testbench, or more generally, in external entity,
        is a definition of expected ports from another entity, for e. g.
        in testbench, component statement contained definition of how the
        tested entity looks like - what are it's expected ports.
  \item process - basic unit of execution in ,,VHDL''.
\end{enumerate}
I also learned the basics about testing - how to create and use testbenches
for created entities and how to analyze tests output - using \textit{gtkwave}.
I've interacted with simple program and it's testbench.

\subsection{Assignment 4}

This problem was a sumup of all previous tasks. The goal here was to create
an entity that using three input ports, called \texttt{A}, \texttt{B} and
\texttt{C} calculated output values \texttt{X} and \texttt{Y} defined as
follows:

$$ x \leftarrow \neg (\neg(a \lor b) \lor (b \lor c)) $$
$$ y \leftarrow (b \lor c) \land \neg(a \oplus c) $$

Created entity had been tested using testbench. A logic table was created
and output generated using created entity was compared to expected values.